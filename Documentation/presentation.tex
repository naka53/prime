\documentclass{beamer}

\mode<presentation> {
\usetheme{Madrid}
}

\usepackage{xcolor}
\usepackage{listings}

\definecolor{mGreen}{rgb}{0,0.6,0}
\definecolor{mGray}{rgb}{0.5,0.5,0.5}
\definecolor{mPurple}{rgb}{0.58,0,0.82}
\definecolor{backgroundColour}{rgb}{0.95,0.95,0.92}

\lstdefinestyle{CStyle}{
    backgroundcolor=\color{backgroundColour},   
    commentstyle=\color{mGreen},
    keywordstyle=\color{magenta},
    numberstyle=\tiny\color{mGray},
    stringstyle=\color{mPurple},
    basicstyle=\footnotesize,
    breakatwhitespace=false,         
    breaklines=true,                 
    captionpos=b,                                         
    showspaces=false,                
    showstringspaces=false,
    showtabs=false,                  
    tabsize=2,
    language=C
}

\title[Linux Kernel Rootkit]{Linux Kernel Rootkit}

\author{Nathan Castets \& Olivier Huge}
\institute[UBX]{Université de Bordeaux}
\date{21 Février 2019}

\begin{document}

\begin{frame}
\titlepage
\end{frame}

\begin{frame}
\frametitle{Overview}
\tableofcontents
\end{frame}

\section{Notions et état de l'art des rootkits}
\subsection{Définitions}

\begin{frame}
\frametitle{Définitions}
\begin{block}{Rootkit}
Utilitaire qui permet d'effectuer différentes actions sur une machine. Le but principal est d'installer un accès privilégié à cette machine pour un pirate de façon persistante dans le temps.
\end{block}
\medskip
A la différence d'un malware classique, le rootkit se veut discret et dissimule au maximum ses actions à l'utilisateur et aux programmes de surveillance.
\end{frame}

\begin{frame}
\frametitle{Définitions}
Il existe 2 types de rootkit :
\begin{itemize}
\item 	Espace utilisateur\\
	Remplace des fonctions utilisées par un programme\\
	Injection de librarie dynamique via \textit{LD\_PRELOAD}
\item	Espace noyau\\
	Remplace des appels systèmes\\
	Module noyau qui écrase la table des appels systèmes
\end{itemize}
\end{frame}

\begin{frame}
\frametitle{Définitions}
\begin{block}{Table des appels systèmes}
Tableau contenant les adresses mémoires des fonctions associées aux appels systèmes. Ces appels systèmes permettent aux programmes de l'espace utilisateur de communiquer avec le noyau.
\end{block}
\medskip
Les appels systèmes sont indispensables pour les programmes de l'espace utilisateur pour utiliser des fonctions que seul le noyau peut exécuter.

On appelle aussi la table des appels systèmes la \textit{sys\_call\_table}.
\end{frame}

\begin{frame}
\frametitle{Définitions}
\begin{block}{KASLR}
La KASLR (Kernel Address Space Layout Randomization) est une sécurité du noyau qui charge aléatoirement les données dans la mémoire.
\end{block}

Cela implique qu'à chaque démarrage du système une structure de donnée n'est généralement pas à la même adresse.

C'est la sécurité principale qui empêche les rootkits de s'installer dans le système.
\end{frame}

\subsection{Pré Linux Kernel 4.17}

\begin{frame}[fragile]
\frametitle{Pré Linux Kernel 4.17}
/fs/open.c :
\begin{lstlisting}[style=CStyle]
/* *** */

EXPORT_SYMBOL(sys_close);

/* *** */
\end{lstlisting}
\medskip
La fonction associée à l'appel système \textit{sys\_close} est accessible par n'importe quel programme présent dans le noyau.

Cet export est présent car le module \textit{mount} a besoin de \textit{sys\_close}.

\medskip
Un brute-force de la mémoire noyau à la recherche des occurences de l'adresse de \textit{sys\_close} nous donne la \textit{sys\_call\_table}.
\end{frame}

\subsection{Post Linux Kernel 4.17}

\begin{frame}
\frametitle{Post Linux Kernel 4.17}
\begin{itemize}
\item 	Suppression de la majorité des appels systèmes dans le code noyau\\
	L'export de la fonction \textit{sys\_close} n'existe plus\\
\item 	Rajout de fonction avec un comportement similaire \textit{ksys\_xyzxyz()}\\
	Le but étant de dissocier au maximum les appels venants de l'espace utilisateur et noyau
\end{itemize}
Cela implique :
\begin{itemize}
\item 	Qu'il n'est plus possible de brute-force la \textit{sys\_call\_table} à l'aide de l'adresse d'un appel système
\item 	Qu'il n'est plus possible d'altérer le comportement de programme présent dans le noyau
\end{itemize}
\end{frame}

\section{Notre Rootkit}
\subsection{Déterminer l'adresse de la table des appels systèmes}

\begin{frame}
\frametitle{Déterminer l'adresse de la table des appels systèmes}
L'idée est de s'intéresser au fonctionnement des appels systèmes et plus précisément au code exécuté en préambule pour préparer l'appel système.

\medskip
Retracer ce code dans la mémoire noyau jusqu'à retrouver un offset vers la \textit{sys\_call\_table}.

\medskip
Nous nous concentrerons sur les version 4.17 à 4.20 du noyau Linux dans la suite de cette présentation.
\end{frame}

\begin{frame}[fragile]
\frametitle{Déterminer l'adresse de la table des appels systèmes}
Dès qu'un appel système est levé, le processeur doit exécuter du code pour préparer cet appel système. L'adresse de ce code se trouve dans le registre \textit{MSR\_LSTAR}. Voyons à l'initialisation ce que contient ce registre.

\medskip
/arch/x86/kernel/cpu/common.c (4.17 - 4.19) :
\begin{lstlisting}[style=CStyle]
if (static_cpu_has(X86_FEATURE_PTI))
	wrmsrl(MSR_LSTAR, SYSCALL64_entry_trampoline);
else
	wrmsrl(MSR_LSTAR, (unsigned long)entry_SYSCALL_64);
\end{lstlisting}

\medskip
/arch/x86/kernel/cpu/common.c (4.20) :
\begin{lstlisting}[style=CStyle]
wrmsrl(MSR_LSTAR, (unsigned long)entry_SYSCALL_64);
\end{lstlisting}
\end{frame}

\begin{frame}[fragile]
\frametitle{Déterminer l'adresse de la table des appels systèmes}
/arch/x86/entry/entry\_64.S (4.17 - 4.20) :
\begin{lstlisting}[style=CStyle]
ENTRY(entry_SYSCALL_64)
	/* *** */
	pushq	%rax

	PUSH_AND_CLEAR_REGS rax=$-ENOSYS
	TRACE_IRQS_OFF

	movq	%rax, %rdi
	movq	%rsp, %rsi
	call	do_syscall_64

	TRACE_IRQS_IRETQ

	movq	RCX(%rsp), %rcx
	movq	RIP(%rsp), %r11

	cmpq	%rcx, %r11
	jne	swapgs_restore_regs_and_return_to_usermode
\end{lstlisting}
\end{frame}

\begin{frame}[fragile]
\frametitle{Déterminer l'adresse de la table des appels systèmes}
/arch/x86/entry/common.c (4.17 - 4.20) :
\begin{lstlisting}[style=CStyle]
__visible void do_syscall_64(unsigned long nr, struct pt_regs *regs)
{
	/* *** */

	nr &= __SYSCALL_MASK;
	if (likely(nr < NR_syscalls)) {
		nr = array_index_nospec(nr, NR_syscalls);
		regs->ax = sys_call_table[nr](regs);
	}

	/* *** */
}
\end{lstlisting}
\end{frame}

\begin{frame}[fragile]
\frametitle{Déterminer l'adresse de la table des appels systèmes}
/arch/x86/entry/common.c (4.17 - 4.20) :
\begin{lstlisting}[style=CStyle]
static __always_inline void do_syscall_32_irqs_on(struct pt_regs *regs)
{
	/* *** */

	if (likely(nr < IA32_NR_syscalls)) {
		nr = array_index_nospec(nr, IA32_NR_syscalls);

	regs->ax = ia32_sys_call_table[nr](regs);

	/* *** */
}
\end{lstlisting}
\end{frame}

\subsection{Patterns pour retrouver l'offset de \textit{sys\_call\_table}}

\begin{frame}[fragile]
\frametitle{Patterns pour retrouver l'offset de \textit{sys\_call\_table}}
Tout d'abord il nous faut l'adresse de la fonction \textit{entry\_SYSCALL\_64} :
\begin{itemize}
\item 	En version 4.20 il nous suffit de lire le registre \textit{MSR\_LSTAR}
\item 	Dans les versions 4.17 - 4.19, on pourrait aussi lire le registre \textit{MSR\_LSTAR} et suivre le code exécuté jusqu'à atteindre \textit{entry\_SYSCALL\_64}
\end{itemize}
\begin{block}{Astuce}
La fonction \textit{native\_load\_gs\_index} qui se trouve juste en dessous de \textit{entry\_SYSCALL\_64} dans le code est exportée via un \textit{EXPORT\_SYMBOL}.
\end{block}
\end{frame}

\begin{frame}[fragile]
\frametitle{Patterns pour retrouver l'offset de \textit{sys\_call\_table}}
Dans \textit{entry\_SYSCALL\_64} on cherche l'appel à \textit{do\_syscall\_64} :
\begin{lstlisting}[style=CStyle]
e8 ?? ?? ?? ??       callq [offset]
\end{lstlisting}
\medskip
Il est précédé par les instructions suivantes :
\begin{lstlisting}[style=CStyle]
4.17 - 4.20
41 57                 push %r15  
45 31 ff              xor %r15d, %r15d  
48 89 c7              mov %rax, %rdi  
48 89 e6              mov %rsp, %rsi  
\end{lstlisting}
\end{frame}

\begin{frame}[fragile]
\frametitle{Patterns pour retrouver l'offset de \textit{sys\_call\_table}}
Dans \textit{do\_syscall\_64} on cherche l'appel à \textit{sys\_call\_table} :
\begin{lstlisting}[style=CStyle]

48 8b 04 fd ?? ?? ?? ?? mov [offset](, %rdi, 8), %rax
\end{lstlisting}
\medskip
Il est précédé par les instructions suivantes :
\begin{lstlisting}[style=CStyle]
4.17  - 4.20
48 19 c0              sbb %rax, %rax  
48 21 c7              and %rax, %rdi  
\end{lstlisting}
\end{frame}

\begin{frame}[fragile]
\frametitle{Patterns pour retrouver l'offset de \textit{sys\_call\_table}}
Dans \textit{do\_syscall\_32\_irqs\_on} on cherche l'appel à \textit{ia32\_sys\_call\_table} :
\begin{lstlisting}[style=CStyle]

48 8b 04 c5 ?? ?? ?? ?? move [offset](, %rax, 8), %rax
\end{lstlisting}
\medskip
Il est précédé par les instructions suivantes :
\begin{lstlisting}[style=CStyle]
4.17  
48 81 fa 81 01 00 00  cmp $0x181, %rdx  
48 19 d2              sbb %rdx, %rdx  
21 d0                 and %edx, %eax  

4.20  
48 81 fa 83 01 00 00  cmp $0x182, %eax  
48 19 d2              sbb %rdx, %rdx  
21 d0                 and %edx, %eax  
48 89 ef              mov %rbp, %rdi  
\end{lstlisting}
\end{frame}

\subsection{Cacher des fichiers à l'utilisateur}

\begin{frame}[fragile]
\frametitle{Cacher des fichiers à l'utilisateur}
Appel système \textit{getdents} :
\begin{lstlisting}[style=CStyle]
asmlinkage long sys_getdents64(unsigned int fd,
				struct linux_dirent64 __user *dirent,
				unsigned int count);
\end{lstlisting}
\medskip
structure \textit{linux\_dirent} :
\begin{lstlisting}[style=CStyle]
 struct linux_dirent {
	unsigned long  d_ino;
	unsigned long  d_off;
	unsigned short d_reclen;
	char d_name[1]; 
}
\end{lstlisting}
\end{frame}

\section{Conclusion}

\begin{frame}[fragile]
\frametitle{Conclusion}
\begin{itemize}
\item 	Technique courante utilisant l'export de la fonction \textit{sys\_close}
\item 	Version 4.17 du noyau rendant cette technique obselète
\item 	Développement d'une technique alternative basée sur la recherche d'un appel à la \textit{sys\_call\_table} dans le code en mémoire
\item 	Exemple de hook de \textit{getdents} pour cacher des fichiers à l'utilisateur
\end{itemize}
\end{frame}

\section{Références}

\begin{frame}[allowframebreaks]
\frametitle{Références}
\begin{thebibliography}{1}
	\bibitem{1}
	Sources du projet\\
	github.com/naka53/prime\\

	\bibitem{1}
	System calls in the Linux kernel\\
	0xax.gitbooks.io/linux-insides/content/SysCall\\

	\bibitem{3}
	Linux Kernel Sources\\
	github.com/torvalds/linux\\
\end{thebibliography}
\end{frame}
\end{document}