\documentclass{beamer}

\mode<presentation> {
\usetheme{Madrid}
}

\title[Linux Kernel Rootkit]{Linux Kernel Rootkit}

\author{Nathan Castets \& Olivier Huge}
\institute[UBX]{Université de Bordeaux}
\date{20 Février 2019}

\begin{document}

\begin{frame}
\titlepage
\end{frame}

\begin{frame}
\frametitle{Overview}
\tableofcontents
\end{frame}

\section{Notions et état de l'art des rootkits}
\subsection{Définitions}

\begin{frame}
\frametitle{Définitions}
\begin{block}{Rootkit}
Utilitaire qui permet d'effectuer différentes actions sur une machine. Le but principal est d'installer un accès privilégié à cette machine pour un pirate de façon persistante dans le temps.
\end{block}
\medskip
A la différence d'un malware classique, le rootkit se veut discret et dissimule au maximum ses actions à l'utilisateur et aux programmes de surveillance.
\end{frame}

\begin{frame}
\frametitle{Définitions}
Il y a 2 types de rootkit :
\begin{itemize}
\item 	Espace utilisateur\\
	Remplace des fonctions utilisées par un programme\\
	Injection de librarie dynamique via \textit{LD\_PRELOAD}
\item	Espace noyau\\
	Remplace des appels systèmes\\
	Module noyau qui écrase la table des appels systèmes
\end{itemize}
\end{frame}

\begin{frame}
\frametitle{Définitions}
\begin{block}{Table des appels systèmes}
Tableau contenant les adresses mémoires des fonctions associées aux appels systèmes. Ces appels systèmes permettent aux programmes de l'espace utilisateur de communiquer avec le noyau.
\end{block}
\medskip
Les appels systèmes sont indispensables pour les programmes de l'espace utilisateur pour utiliser des fonctions que seul le noyau peut exécuter.

On appelle aussi la table des appels systèmes la \textit{sys\_call\_table}.
\end{frame}

\subsection{Pré Linux Kernel 4.17}

\begin{frame}
\frametitle{Pré Linux Kernel 4.17}

\end{frame}

\subsection{Post Linux Kernel 4.17}

\begin{frame}
\frametitle{Post Linux Kernel 4.17}
\end{frame}

\section{Notre Rootkit}
\subsection{Déterminer l'adresse de la table des appels systèmes}

\begin{frame}
\frametitle{Déterminer l'adresse de la table des appels systèmes}
\end{frame}

\subsection{Hook un appel système}

\begin{frame}
\frametitle{Hook un appel système}
\end{frame}

\subsection{Cacher des fichiers à l'utilisateur}

\begin{frame}
\frametitle{Cacher des fichiers à l'utilisateur}
\end{frame}

\section{Conclusion}

\begin{frame}
\frametitle{Conclusion}
\end{frame}

\end{document}